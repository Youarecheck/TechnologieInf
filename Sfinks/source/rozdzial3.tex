\section{Chrystus królem i obrońcą Polski}

Ten rozdział uważam za najistotniejszy, gdyż dotyczy kluczowych momentów w
historii Polski, kiedy jej istnienie było poważnie zagrożone, a pomoc Opatrzności Bożej
wydawała się najbardziej namacalna. Będę analizował chwile, gdy zarówno naród, jak i
władcy z pełnym zaufaniem powierzali Polskę opiece Najświętszej Maryi Panny. Przyjrzę
się zwycięstwom, upadkom oraz duchowej sile, która inspirowała do działania. W rozdziale
tym nie zabraknie refleksji nad wydarzeniami takimi jak obrona Jasnej Góry w czasie
potopu szwedzkiego czy śluby lwowskie Jana Kazimierza, które łączyły duchowość z
polityką. Rozdział ten pokaże, jak wiara w Chrystusa Króla oraz opieka Matki Bożej
stanowiły oparcie dla narodu w najtrudniejszych chwilach jego dziejów. Jednakże najpierw
chciałbym wykazać pewne podobieństwa Polski do narodu wybranego Izraela.

\subsection{Historia Izraela a Polska: Wiara i losy narodów}

Na wstępie pragnę zaznaczyć, że celem tego podrozdziału nie jest sugerowanie, iż
Polska i Izrael mają bezpośrednie powiązania historyczne czy genealogię. Chciałbym
jednak zwrócić uwagę na pewne podobieństwa w losach obu narodów, szczególnie w
kontekście ich relacji z Bogiem i przestrzegania Jego przykazań. Oparłem się na biblijnej
wersji historii Izraela, która ukazuje duchowe fundamenty tego narodu oraz ich
konsekwencje w wymiarze narodowym i duchowym.

\subsubsection{Początek Izraela}

Historia Izraela, zgodnie z Księgą Rodzaju, ma swoje korzenie w obietnicach Boga
wobec Abrahama, którego uważa się za „ojca wiary”. Bóg przemienił imię Jakuba na Izrael
(Rdz 32,29), co oznacza „ten, który walczy z Bogiem”. Z dwunastu synów Jakuba wywodzą
się dwanaście plemion Izraela, które założyły państwo w ziemi Kanaan – ziemi obiecanej
Abrahamowi i jego potomkom. Narodziny Izraela były więc związane z obietnicami Bożymi,
wiarą i wiernością przodków.

\subsubsection{Narodziny Polski}

Podobnie, narodziny Polski symbolicznie wiążą się z chrztem Mieszka I. Przyjęcie
chrześcijaństwa w 966 roku było aktem poddania się pierwszego władcy Polski pod
zwierzchnictwo Boga. Podobnie jak w przypadku Izraela, fundamentem duchowym 
narodzin Polski była wiara przywódcy, która miała wpływ na kształtowanie tożsamości
całego narodu.

\subsubsection{Upadki narodów}

Następnym kluczowym podobieństwem myślę, że jest podobny cykl wzrastania
narodu i państwa oraz jego upadania. W Piśmie Świętym, szczególnie w Starym
Testamencie, łatwo zauważyć, że gdy naród lub jego przywódcy odwracali się od Boga,
prowadziło to do Jego niełaski i katastrofalnych konsekwencji łącznie z takimi jak utracenie
Państwa i bycie zmuszonym na tułaczkę i ramach innych Państw lub mocarstw.
Zacznijmy od przykładów z Królestwa Judy. Jednym z nich jest niewola babilońska
która wydarzyła się w VI wieku p.n.e. Niewola babilońska była wynikiem upadku Królestwa
Judy w 586 roku p.n.e., kiedy to Babilończycy pod wodzą Nabuchodonozora II zdobyli
Jerozolimę, zburzyli świątynię i uprowadzili mieszkańców do Babilonii. Upadek Izraela
przepowiedział Bóg przez proroka Jeremiasza ostrzegając przed konsekwencjami
nieposłuszeństwa i bałwochwalstwa. Proroctwo zapowiada 70-letnią niewolę babilońską
jako karę za grzechy.3 Natomiast 2 Księga Królewska 24:20 i 25:1-21 – Opisuje ostateczne
oblężenie i zniszczenie Jerozolimy. Podkreślone jest, że to odstępstwo od przymierza z
Bogiem doprowadziło do upadku. Biblia wymienia liczne grzechy, takie jak
bałwochwalstwo (kult Baala, Aszery), niesprawiedliwość społeczna i ignorowanie
proroków.
Innym znanym upadkiem Izraela było podbicie przez Asyrię 722 roku p.n.e. W tym to
właśnie roku Królestwo Izraela (północne) zostało podbite. Po zdobyciu stolicy, Samarii,
ludność została przesiedlona do różnych części imperium asyryjskiego. Księga Królewska
17:7-23 W sposób szczegółowy opisuje przyczyny upadku. Podkreślane są
bałwochwalstwo, brak posłuszeństwa wobec Prawa Mojżeszowego oraz ignorowanie
ostrzeżeń proroków. Prorok Ozeasza piętnuje naród za odejście od Boga i oddanie się
grzechowi. 4

W obu przypadkach widoczny jest związek między upadkiem duchowym a
polityczną katastrofą. Biblia podkreśla, że odejście od Boga, ignorowanie Jego Prawa i
oddawanie czci innym bogom prowadziło do osłabienia narodowego i w końcu do jego
upadku. Zarówno niewola babilońska, jak i upadek Królestwa Izraela są tego wyraźnym
świadectwem.
Mając na uwadze fragmenty historii Izraela, przejdźmy do Polski. W XVI wieku ruch
reformacyjny dotarł do Polski, wywołując istotne zmiany w sferze religijnej. Kalwinizm, 
luteranizm oraz ruch braci polskich (arian) zdobyły licznych zwolenników, szczególnie
wśród szlachty i magnaterii. W wyniku tego katolicka dominacja osłabła, a w niektórych
regionach kościoły były przejmowane przez protestantów. Spory religijne prowadziły do
wewnętrznych napięć, co osłabiło jedność religijną, mając wpływ na destabilizację
państwa.

Reformacja przyczyniła się do osłabienia pozycji Kościoła katolickiego, który pełnił
także funkcję integracyjną. Polska znalazła się na progu wewnętrznych konfliktów, co
miało swoje konsekwencje w XVII wieku. Jednakże spadek religijności i osłabienie pozycji
Kościoła nie były jedynymi czynnikami prowadzącymi do osłabienia państwa. Znaczącą
rolę w tym procesie odegrała także magnateria, która stała się skorumpowana i coraz
mniej angażowała się w sprawy narodowe. Taki stan rzeczy pokazuje, jak brak moralnych
fundamentów w rządzących warstwach władzy może prowadzić do destabilizacji państwa.
W XVIII wieku Rzeczpospolita Obojga Narodów pogrążyła się w anarchii politycznej i
ekonomicznej. Słabość państwa była wynikiem m.in. liberum veto, oligarchizacji władzy
oraz interwencji państw ościennych (Rosji, Prus, Austrii). Jednocześnie nastąpiło
osłabienie moralnego autorytetu Kościoła. W okresie oświecenia wielu przedstawicieli elit
przyjęło idee deizmu, materializmu i racjonalizmu, które osłabiały rolę religii. Reformy
Komisji Edukacji Narodowej, choć pozytywne z punktu widzenia oświaty, zredukowały
wpływ Kościoła na szkolnictwo. Osłabienie pozycji Kościoła i spadek religijności wśród elit
wpłynęły na kryzys tożsamości narodowej. W rezultacie państwo stopniowo upadło, co
zakończyło się rozbiorami w 1772, 1793 i 1795 roku.
Mając oba konteksty historyczne wzięte pod uwagę wyraźnie widać jak spadek wiary
i ufności w Boga pośród ludności i elit poprzedzało i szło w parze wraz z kryzysami i
katastrofami dotykającymi Państwo. Czy to przypadek? Być może, lecz zobaczmy czy
podobny efekt idzie w drugą stronę. Czy umacnianie się wiary i kultu Bożego miało
pozytywny wpływ na Państwo i naród. 

\subsubsection{Odrodzenie wiary i Państwa}

Okres niewoli Babilońskiej był bardzo ciężkim okresem dla ludu Królestwa Judy,
gdyż zmagał się z wielkim uciskiem. Jednakże w tych trudnych monetach niewątpliwie
wiara stanowo wiła podporę i nadzieję. Warto zaznaczyć, iż ta wiara przetrwała dzięki
interwencji Króla Jozjasza który usłyszawszy słowa księgi Prawa Następnie zebrał
starszyznę Judy i cały lud w świątyni, odczytał słowa Prawa i odnowił przymierze z Panem

(2 Księga Królewska 22:8-11, 23:1-3). Niestety upadek Królestwa był nie unikniony, ale to
wydarzenie umocniło naród na tyle by przetrwać to co miało się wydarzyć. Powrót
Izraelitów z niewoli babilońskiej pod panowaniem perskiego króla Cyrusa Wielkiego
stanowi istotny moment odnowy duchowej i narodowej. Cyrus, w duchu tolerancji
religijnej, wydał edykt pozwalający Żydom wrócić do Jerozolimy i odbudować świątynię.
Przykład ten ukazuje, jak powrót do wiary w Boga i odbudowa kultu świątynnego wzmocniły
tożsamość narodową Izraela. Księga Ezdrasza (Ezd 1:1-4) opisuje zarówno edykt Cyrusa,
jak i zaangażowanie w odbudowę świątyni, które było wyrazem wierności wobec
przymierza z Bogiem.

Pewne podobieństwo odnajduję w historii Polski, która przez odsunięcie się od
Boga upadła i utraciła swą niepodległość. Jednakże wiara nie została całkowicie wytępiona
w narodzie i tak jak w przypadku Izraela tak i Polsce odegrała ważną rolę w przetrwaniu
narodu. W czasach zaborów Kościół katolicki odegrał kluczową rolę w podtrzymywaniu
polskiej tożsamości narodowej i duchowej. Szczególnie w zaborze rosyjskim i pruskim
wiara katolicka była istotnym elementem oporu wobec rusyfikacji i germanizacji. Przykład
ten uwidacznia się w działalności duchowieństwa, które wspierało ruchy
niepodległościowe, np. w powstaniach listopadowym i styczniowym. Po 123 latach
rozbiorów, odzyskanie niepodległości było nie tylko triumfem politycznym, ale także
duchowym odrodzeniem narodu, który przetrwał dzięki wierze i tradycji katolickiej.
Symbolicznym wyrazem odnowy duchowej była obecność religii w życiu publicznym
odrodzonego państwa oraz zaangażowanie Kościoła w budowanie nowej Polski.


Kończąc powoli ten podrozdział chcę wspomnieć, iż przypadków, kiedy to kościół
odgrywał w Polsce znaczącą rolę w przetrwaniu tożsamości narodu i jego przetrwaniu było
znacznie więcej. Przykładowo w okresie PRL, kiedy próbowano złamać ducha w Polakach i
w pewien sposób podać całkowicie kontroli Rosjan oraz doktrynie komunistycznej. Kościół
jednak stawiał temu opór i motywował naród do walki o swe wartości i wolność przez co
sam stawał się celem który komuniści usiłowali zniszczyć. Warto przy tym wspomnieć
kilku bohaterów, takich jak Karol Wojtyła (Papież Polak Św. Jan Paweł II), Kardynał Stefan
Wyszyński, ksiądz Popiełuszko. Oni wszyscy zostali zapamiętani jako Polscy patrioci,
którzy ukochali swą Ojczyznę i walczyli o nią. Może nie koniecznie pistoletem i szablą, ale
Słowem i nieustraszoną postawą. Historia zarówno Izraela, jak i Polski pokazuje, że
fundamenty duchowe i wierność wierze były niezastąpione w przetrwaniu i odbudowie
narodów. Wierność Bogu nie tylko kształtowała ich tożsamość, ale także dawała siłę w
najtrudniejszych chwilach, pozwalając zachować jedność i nadzieję na odrodzenie.

\subsection{Jezus Chrystus Królem Polski i Najświętsza Maryja Panna Królową
Polski.}

W poprzednich rozdziałach opisywałem znaczenie decyzji pierwszego Króla Polski
co do chrztu i przyjęcie Trójcę Świętą za jedynego Boga. Tamten moment poniósł się
echem przez wszystkie dzieje Polski. Jednakże ktoś mógł powiedzieć, czy na podstawie
jednego decyzji pojedynczego człowieka nawet Króla można tę przynależność przypisywać
całemu narodowi? Nie, lecz to na co zwrócę uwagę, iż podanie się Polaków pod władzę
Chrystusową i opiekę opatrzności Bożej nie była decyzją pojedynczego monarchy, lecz
decyzją wielu monarchów oraz narodu powtarzaną wiele krotnie na przestrzeni wieków i
stuleci. Oto krótka lista ów wydarzeń:

\begin{itemize}
    \item Śluby Lwowskie (1656) – Król Jan Kazimierz złożył śluby narodowe,
    powierzając Polskę opiece Najświętszej Maryi Pannie, prosząc ją o ratunek w
    czasie kryzysu wojennego, zwłaszcza podczas potopu szwedzkiego. Śluby
    miały miejsce na Jasnej Górze, gdzie obraz Matki Bożej Częstochowskiej
    został uznany za obrończynię Polski.
    \item Ślubowanie młodego państwa polskiego (1918) – Po odzyskaniu
    niepodległości przez Polskę w 1918 roku, rząd i ludzie złożyli akt poświęcenia
    Polski Bogu, modląc się o pomyślność młodego państwa, co stało się
    symbolicznym wyrazem zawierzenia przyszłości kraju w ręce boskie.
    \item Cud nad Wisłą (1920) – Podczas wojny polsko-bolszewickiej, w obliczu
    zbliżającej się inwazji bolszewickiej, wierzący Polacy modlili się do
    Najświętszej Maryi Panny, a zwycięstwo pod Warszawą zostało uznane za
    cud. Działania wojenne były ściśle związane z wiarą, a Polacy uznali
    zwycięstwo za interwencję boską.
    \item Poświęcenie Polski Najświętszemu Sercu Jezusa (1920) – W czasie wojny
    polsko-bolszewickiej, po zwycięstwie w bitwie warszawskiej, Polska została
    poświęcona Najświętszemu Sercu Jezusowemu, co miało na celu ochronę
    kraju przed dalszymi zagrożeniami zewnętrznymi.
  \end{itemize}


  Takich wydarzeń było więcej, lecz aby nie przepełniać dokumentu wybrałem tylko
kilka. Chciałbym, jednakże wyróżnić kilka wydarzeń szczególnie mi bliskich.

\textbf{Obrona Jasnej Góry}: Jasna Góra była broniona przez żołnierzy, mnichów
oraz ludność cywilną. Dowódcą obrony był przeor klasztoru, o. Augustyn Kordecki,
który z wielką determinacją organizował obronę przed wiele krotnie liczniejszymi
szwedzkimi wojskami. Obrona Jasnej Góry trwała od 18 listopada do 27 grudnia
1655 roku. Szwedzi, pomimo przewagi liczebnej, nie byli w stanie zdobyć klasztoru. 

Obrona była możliwa dzięki m.in. dobrze przygotowanym umocnieniom,
sprzyjającemu terenowi oraz determinacji obrońców. W trakcie obrony modlitwy,
post i pokuta odgrywały kluczową rolę w mobilizacji duchowej obrońców. Modlitwy
o pomoc i interwencję Matki Boskiej Częstochowskiej miały szczególne znaczenie,
ponieważ Jasna Góra była uważana za miejsce cudowne, gdzie znajdował się słynny
obraz Matki Bożej Częstochowskiej. Obrona Jasnej Góry została uznana za
cudowną, ponieważ po wielu dniach intensywnych walk i prób zdobycia klasztoru,
Szwedzi zostali zmuszeni do odwrotu. Po obronie Jasnej Góry morale Polaków
wzrosły, a wydarzenie to zostało uznane za punkt zwrotny w Potopie Szwedzkim,
mobilizując Polaków do dalszego oporu. Po zwycięstwie pod Częstochową Polacy
szli za ciosem i odnieśli potężną serię zwycięstw. 

\textbf{Jasnogórskie Śluby Narodu Polskiego: 26}sierpnia 1956 na Jasnej Górze przy
udziale około według znanych mi źródeł od ok. 200 tys. do kilku milionów.
wiernych. Rotę Ślubów odczytał bp Michał Klepacz, pełniący obowiązki
przewodniczącego Episkopatu Polski. Autorem tych odnowionych ślubów był
Kardynał Stefan Wydrzyński, który przebywał w tamtym momencie w więzieniu
między innymi za to, że nie chciał podporządkować się komunistą i Rosjanom.
Walczył on o Polskę oraz za kościół. Stanowi on przykład prawdziwego Polskiego
patrioty i silnego pasterza.

Te wydarzenia doskonale pokazują, że Boga w Trójcy Świętej dla Polski nie
wybrał tylko jeden człowiek, ale sam naród wybrał Chrystusa Pana Panów i Króla
Królów jako władcę tego wspaniałego narodu. To wola prostego ludu wybrała Matkę
Bożą na królową Polski, naszą Orędowniczkę i Matkę.

\subsection{Podsumowanie}

Tak oto w zgrabny sposób przeszliśmy przez ważne i kluczowe wydarzenia
historyczne Polski. Wykazałem powiązania między wiarą, a stanem rzeczywistym Państwa.
Ukazałem piękno wypływające ze piękna i kreatywności płynące z inspiracji dobrem
Najwyższym. Polska ma ślinie związaną swą historię i fundamenty z kościołem katolickim.
Jest to wynikiem zarówno natury jej narodzin jak i wiele krotnego aktów woli Przywódców i
ludu. Głównym celem tej pracy nie było nawracanie kogo kolwiek lecz ukazanie piękna
historii Polski oraz ukazanie fundamentalnych podstaw Państwa Polskiego. Tego co leży w
głębi tego zacnego narodu. W dzisiejszych czasach próbuje się odbierać narodom ich
tożsamość 
